\documentclass[assignment08_Solutions]{subfiles}

\invalidatemargin

%\IfSubStr{\jobname}{\detokenize{Solutions}}{\toggletrue{solutions}}{\toggletrue{solutions}}

\IfSubStr{\jobname}{\detokenize{Solutions}}{\toggletrue{solutions}}{\togglefalse{solutions}}

\fancypagestyle{firstpage}

{\rhead{Assignment 8 \linebreak \textit{Version: \today}}}

\title{Assignment 8: Computer Vision Project Preparation}
\author{Machine Learning}
\date{Fall 2021}

\begin{document}

\maketitle
\thispagestyle{firstpage}

Note: Parts of this assignment are due early, so you can build off each others' ideas. The project pitches are due Thursday morning, so we have time to print and organize them.
\vspace{2em}



\begin{learningobjectives}
\bi
\item Explore potential applications of machine learning through reading about the work of others
\item Reflect on course experiences and draft personal learning goals
\item Ideate computer vision applications for your project
\ei
\end{learningobjectives}


%
%\section{Read About Evaluation of ML Models (80 minutes)}
%
%\begin{exercise}[(80 minutes)]
%Evaluating your model within the context of your application will be an important component of this project.  We've seen some basic techniques for model validation, but they are far from the whole story.  We thought that doing some readings on how to think about evaluating ML systems as well as learning about some common ML pitfalls would provide helpful framing for the projects you all come up with.
%
%\bes
%\item \href{https://www.youtube.com/watch?v=tleeC-KlsKA}{Watch Machine Learning Gremlins} (30 minutes)
%\item Read one of the following papers and be prepared to discuss the main ideas in class.
%\bi
%\item \href{https://towardsdatascience.com/breaking-neural-networks-with-adversarial-attacks-f4290a9a45aa}{Breaking NNs with Adversarial Attacks}
%%\item \href{http://danielnee.com/2015/01/common-pitfalls-in-machine-learning/}{Common Pitfalls in ML} (this is pretty similar to the gremlins video, but consider this if you want to reinforce these points). Broken links
%\item \href{https://dl.acm.org/citation.cfm?id=3287598}{Fairness and Abstraction in Sociotechnical Systems} (for a view on system evaluation that takes a society technology and society approach).  This one is pretty dense, but worth the read.
%\ei
%\ees
%
%\end{exercise}


\section{Develop Project ideas}

\begin{exercise}[(20 minutes) Due Tuesday night]
Prepare for project brainstorming.
\bes
\item Read the \href{https://github.com/olincollege/MLfall2021/blob/master/Module\%201/m1_project/m1_project.pdf}{project description}.
\item Draft your own learning goals for this project. Be sure to bring these to class.
\item Explore the suggested data sets (in the project description). You are also welcome to seek your own data set, though this may require additional time.
\item Come up with at least 5 potential applications for computer vision, and share them here (there is a separate tab for morning and afternoon sections): \href{https://docs.google.com/spreadsheets/d/1TyzKsfdCvZEzfaYswfJSHiYH2JU73gd0KCIOwQijvFo/edit?usp=sharing}{Google Sheet for Ideas} 
\ees 
\end{exercise}
 
\vspace{1em}
We hope that reviewing what other people post will help you to come up with even more creative ideas, and to get a sense of what others might be excited about. We have found that selecting a few project ideas and thinking about them in more detail can be useful in finding good pairings for partners. We ask you to be sure to complete the next exercise by Thursday at 8 am, so that we can print and cluster your project ideas to make better use of class time.
\vspace{1em}
\begin{exercise}[(60 minutes) Due by Thursday at 8 am]
\bes
\item Review some of the ideas that others' shared in the \href{https://docs.google.com/spreadsheets/d/1TyzKsfdCvZEzfaYswfJSHiYH2JU73gd0KCIOwQijvFo/edit?usp=sharing}{Google Sheet for Ideas} 
\item Select at least two potential applications that you find compelling, and flesh them out. You may want to consider some of the prompts in the Appendix A of the project description.
\item Create a pitch slide for both ideas in our shared slide deck. Please make a copy of the template slide, and be sure to keep the template as the first slide so others can find it. We will share these in the next class. \href{https://docs.google.com/presentation/d/1f3u9lJXV9_DzKRhHH41kr8h3xQn4ex6SbpcNwyws9z4/edit?usp=sharing}{Slide deck and template.} 
\ees
\end{exercise}

\end{document}
