\documentclass[assignment03_Solutions]{subfiles}

\IfSubStr{\jobname}{\detokenize{Solutions}}{\toggletrue{solutions}}{\togglefalse{solutions}}

\fancypagestyle{firstpage}

{\rhead{Assignment 3 \linebreak \textit{Version: \today}}}

\title{Assignment 3: Probabilistic Definitions of Fairness and Text Classification}
\author{Machine Learning}
\date{Fall 2019}
\invalidatemargin

\begin{document}

\maketitle
\thispagestyle{firstpage}


\begin{learningobjectives}
\bi
\item Learn about the connection between probabilistic criteria for algorithmic fairness and Bayesian Networks.
\item Solidify your understanding of the Na\"ive Bayes algorithm by applying it to movie review sentiment classification.
\ei
\end{learningobjectives}

\section{Bayesian Networks and Algorithmic Fairness}


In assignment 1 of this module we discussed how Bayesian methods can be used to reason about algorithmic fairness.  We've just had some lengthy discussions about fairness within the context of the COMPAS algorithm.  We touched upon some of the limitations of statistically based notions of fairness.  Nevertheless, these criteria do have a potential role to play, and you should know what the most common definitions of fairness are and what assumptions they make.

As context for the reading and to help us have common notation, suppose we have the following random variables.

\bi
\item $R$ represents the prediction generated by our algorithm.
\item $A$ represents a sensitive attribute
\item $Y$ represents the thing we're trying to predict (we want $R = Y$ if we are accurate)
\ei

\begin{externalresources}[(40 minutes)]
Read \href{https://fairmlbook.org/classification.html}{Fairness and Machine Learning Chapter 2}.  Start at the section \emph{Formal non-discrimination criteria} and read up to (but not including) the section \emph{Calibration and sufficiency}.
\begin{notice}
\bi
\item Don't get too hung up on the \href{https://en.wikipedia.org/wiki/Receiver_operating_characteristic}{ROC curves}.  We can discuss this on NB, but it is not required to understand what is going on here.  If you decide to check it out, you'll see an example of an ROC curve in the notebook linked below (it is optional).
\item The notation they use in this reading for conditional independence is $\bot$ (instead of our notation, $\condindep$).
\ei
\end{notice}
\end{externalresources}


\begin{exercise}[(10 minutes)]
Thinking back to the COMPAS example, which definition of fairness given in the reading was ProPublica using?  Which definition of fairness was Northpointe using?

\begin{boxedsolution}
\bi
\item Northpointe is using sufficiency $Y \condindep A~|~R$.  You'll notice that in the reading they say that sufficiency is the same thing as matching positive and negative predictive value for all values of the protected attribute.
\item Propublica is using separation $R \condindep A~|~Y$.  This fairness principle requires the false positive and true positive rates to be the same across for all values of the protected attribute.
\ei
\end{boxedsolution}
\end{exercise}

%If you're interested in examining the Broward County COMPAS data yourself, we have put together \href{https://colab.research.google.com/github/mlfa19/assignments/blob/master/Module\%202/03/Exploring_COMPAS_Data.ipynb}{a notebook that reproduces the calculations that are at the heart of the two competing notions of fairness}.  The notebook is written in with some notebook exercises, but these are totally optional.  We expect that the default will be that folks will just take a look if they want to (or skip this if they don't). If you want to spend some extra time and solidify your understanding of true positive rate, false positive rate, positive predictive value, etc., then it might be worth doing it as a set of exercises.

For the purposes of posting on NB, the conversion of the notebook is at the end of the document.

\section{Text Classification with Bag of Words}
Next we'll be applying Na\"ive Bayes to the task of classifying text.

\begin{externalresources}[(60 minutes)]
This will be done in the \href{https://colab.research.google.com/github/mlfa19/assignments/blob/master/Module\%202/03/Assignment_3_Companion_Notebook.ipynb}{Assignment 3 companion notebook}.
\end{externalresources}

(to make this more readable, we have included the PDF of the companion notebook at the end).

%\companionnotebook{Exploring_COMPAS_Data}


\companionnotebook{Assignment_3_Companion_Notebook}


\end{document}
